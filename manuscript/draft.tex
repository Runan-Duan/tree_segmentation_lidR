\documentclass{isprs} % isprs class modified 23-04-2019 (Dennis Wittich)
\usepackage{subfigure}
\usepackage{setspace}
\usepackage{geometry} % added 27-02-2014 Markus Englich
\usepackage{epstopdf}
\usepackage[labelsep=period]{caption}  % added 14-04-2016 Markus Englich - Recommendation by Sebastian Brocks
\usepackage[british]{babel} 
\usepackage[hang]{footmisc}
\def\footnotemargin{1em} % added 08-01-2020 Dennis Wittich

%\usepackage[authoryear]{natbib}
%\def\bibhang{0pt}
\geometry{a4paper, top=25mm, left=20mm, right=20mm, bottom=25mm, headsep=10mm, footskip=12mm} % added 27-02-2014 Markus Englich
%\usepackage{enumitem}
%\usepackage{isprs}
%\usepackage[perpage,para,symbol*]{footmisc}
%\renewcommand*{\thefootnote}{\fnsymbol{footnote}}
\captionsetup{justification=centering,font=normal} % thanks to Niclas Borlin 05-05-2016
\captionsetup[figure]{font=small} % added 23-04-2019 Dennis Wittich
\captionsetup[table]{font=small} % added 23-04-2019 Dennis Wittich

\begin{document}

\title{Automated Individual Trees Segmentation workflow on different forest: parameters exploration}
\date{}

\author{
 Runan Duan\textsuperscript{1}, Ian Dowman\textsuperscript{2}}

\address{
	\textsuperscript{1 }Institute of Geography, Heidelberg University, Im Neuenheimer Feld 368, Heidelberg, 69120, Baden-W ̈urttemberg, Germany - (runan.duan)@stud.heidelberg-de\\
}

\abstract{
Abstract}

\keywords{Tree Segmentation, PDAL, PyTreeDB}

\maketitle

\section{Introduction}
 Relevance of the topic in general and in geography
 – Scope and definition of the specific topic
 – Objective of the specific paper / work → small research question!














\sloppy

\section{State of the Art}
– Current aspects of the topic under research
– Overview about different methods
– Related applications examples

\subsection{}

\subsection{}














\section{Methods and Data}
– Description of the dataset used in the analysis
– Detailed explanation of the used methods and algorithm: 
→ How does it work?
 → What are input and output data?
 → Which parameters are required, which settings are used?

















\section{Results}
– Results of the analysis (as presented in section 3)




































\section{Discussion and Conclusion}
– Discussion of results in the context of objective and state of the art
– Most important findings of the analysis and regarding the method



















{
	\begin{spacing}{1.17}
		\normalsize
		\bibliography{} % Include your own bibliography (*.bib), style is given in isprs.cls
	\end{spacing}
}


































































































\newpage

%\itemize
\begin{enumerate}
\setlength\itemsep{0em}\setlength\parskip{0em}\setlength\topsep{0em}\setlength\partopsep{0em}\setlength\parsep{0em} 
\item{Title of the paper} 
\item{Author(s) and affiliation, \textbf{anonymized} for the double-blind review process}
\item{Keywords (max. 6)}
\item{Abstract (100--250 words)}
\item{Introduction}
\item{Main body}
\item{Conclusions}
\item{Acknowledgements,  \textbf{anonymized} for the double-blind review process}
\item{References}
\item{Appendix (if applicable)}
\end{enumerate}


% In Section~\ref{MANUSCRIPT} we present related work
%\newpage            
\subsection{Page Layout and Length}\label{sec:Page Layout, Spacing and Margins}
The paper must be compiled in one column for the title, author information, keywords and abstract and in two columns for all subsequent text. Left and right justified typing is mandatory. All manuscripts, except Invited Papers are limited to a length of approximately 8 single-spaced pages (A4 size), including abstracts, figures, tables and references. ISPRS Invited Papers are limited to approximately 12 pages. In any case, the minimum length of any paper is 6 pages. The font type Times New Roman with a size of 9 pts. is to be used.

% KAO: Removed spacing before label: can cause references to be wrong
\begin{table}[h]
	\centering
		\begin{tabular}{|l|c|c|}\hline
			Setting&\multicolumn{2}{c|}{A4 size page}\\\hline
			  &mm&inches\\
			 Top&25&1.0\\
			 Bottom&25&1.0\\
			 Left&20&0.8\\
			 Right&20&0.8\\
			 Column Width&82&3.2\\
			 Column Spacing&6&0.25\\\hline
		\end{tabular}
	\caption{Margin settings for A4 size page.}
\label{tab:Margin_settings}
\end{table}

\subsection{Style Guides}\label{sec:Preparation in electronic form}

% KAO: Remove newline
To assist authors in preparing their contributions, style guides are provided in Word and LaTeX on the ISPRS web site, see www.isprs.org/documents/orangebook/app5.aspx. Use of these style guides ensures that the paper is correctly formatted and is therefore strongly suggested.


\section{Title and Abstract Block}\label{sec:TITLE AND ABSTRACT BLOCK}

\subsection{Title}\label{sec:Title}

The title must appear centred in bold at the top of the first page with a size of 12 pts. Any author information must be masked out in the full paper submitted for double-blind peer review.

Author(s) name(s), affiliation and mailing address only appear in the camera-ready manuscript, which is to be submitted after the paper is accepted for publication. For this camera-ready manuscript, type the full author(s) name(s), affiliation and mailing address (including e-mail), centred under the title. In the case of multi-authorship, indicate which author belongs to which organisation.


% the following subsection was deleted, Sept. 20, 2022
%\subsection{ISPRS Affiliation (optional)}\label{sec:ISPRS Affiliation (optional)}

% KAO: Use proper quotes
%For those authors affiliated with a specific Commission and/or Working Group of 
%ISPRS, a separate title may be entered. The title should be centred in bold type 
%after one blank line below the author's affiliation, i.e. Commission~\#, Working Group~\#. 
%The Commission number shall be Roman and the Working Group number should be the Commission 
%Roman number, slash, WG Arabic number (see example above).


\subsection{Keywords}\label{sec:Keywords}

% KAO: Use proper quotes and dash
Leave two lines blank, then type \textbf{``Keywords:''} in bold, followed by a maximum of 6 keywords. Note that ISPRS does not provide a set list of keywords. Therefore, include those keywords which you would use to find a paper with content you are preparing.


\subsection{Abstract}\label{sec:Abstract}

% KAO: Use proper quotes and dash
Leave two blank lines under the keywords. Type \textbf{``Abstract''}
flush left in bold Capitals followed by one blank line. Note that the abstract should be concise (100 - 250 words), present briefly the content and very importantly, the scientific contribution and results of the paper in words understandable also to non-specialists.


\section{Main Body of Text}\label{sec:MAIN BODY OF TEXT}

Type text single-spaced, \textbf{with} one blank line between paragraphs and 
following headings. Start paragraphs flush with left margin.


\subsection{Headings}\label{sec:Headings}

% KAO: Remove explicit newlines in this section

Major headings are to be numbered, centred in bold, preceded and followed by a blank line.

Subheadings are to be numbered, typed in bold flush with the left margin, preceded and followed by a blank line. 

Subsubheadings are to be numbered, typed in bold after one blank line flush with the left margin, with text following on the same line. Subsubheadings may be followed by a period or colon, they may also be the first word of the paragraph's sentence.

Use decimal numbering for headings, subheadings and subsubheadings.


\subsection{Footnotes}\label{sec:Footnotes}

Mark footnotes in the text with a number (1); use consecutive numbers for following footnotes. Place footnotes at the bottom of the page, separated from the text above it by a horizontal line.


\subsection{Figures and Tables}\label{sec:Illustrations and Tables}

\subsubsection{Placement:}\label{sec:Placement}

Figures must be placed in the appropriate location in the document, as close as practical to the reference of the figure in the text. While figures and tables are usually aligned horizontally on the page, large figures and tables can be rotated by 90 degrees. If so, make sure that the top is always on the left-hand side of the page.

\subsubsection{Captions:}\label{sec:Captions}

All captions should be centred directly beneath the illustration. Use single spacing if they 
use more than one line. All captions are to be numbered consecutively, 
e.g. Figure~1, Figure~2, Figure~3, ..  and Table~1, Table~2, Table~3, ...

% KAO: Remove spacing before label: can cause reference to be wrong
\begin{figure}[ht!]
\begin{center}
		\includegraphics[width=1.0\columnwidth]{figures/test_sites/fig1.eps}
	\caption{Figure placement and numbering.}
\label{fig:figure_placement}
\end{center}
\end{figure}


\subsubsection{Copyright:}\label{sec:Copyright}

% KAO: Inter-sentence spacing
If your article contains any copyrighted illustrations or imagery, 
include a statement of copyright such as: \copyright~SPOT Image Copyright 20xx 
(fill in year), CNES\@. It is the author's responsibility to obtain any necessary 
copyright permission. After publication, your article is distributed under \underline{the Creative 
Commons Unported License} and you retain the copyright.


\subsection{Equations, Symbols and Units}\label{sec:Equations, Symbols and Units}

\subsubsection{Equations:}\label{sec:Equations}

Equations should be numbered consecutively throughout the contribution. The equation 
number is enclosed in parentheses and placed flush right. Leave one blank lines 
before and after equations: 


\begin{equation}\label{equ:1}
	x = x_0 -c \frac{X - X_0}{Z - Z_0}; y = y_0 -c \frac{Y - Y_0}{Z - Z_0},
\end{equation}

\begin{tabbing} 
where \hspace{0.6cm} \= $c$ = principle distance\\
\> $x,y$ = image coordinates\\
\> $X_0,Y_0, Z_0$ = coordinates of projection centre\\
\> $X, Y, Z$ = object coordinates
\end{tabbing}

\subsubsection{Symbols and Units:}\label{sec:Symbols and Units}
Use the SI (Syst\`{e}me International) Units and Symbols. Unusual characters 
or symbols should be explained in a list of nomenclature.

% KAO: Non-breaking space
\subsection{References}\label{sec:References}
References must be cited in the text, thus~\cite{smith1987rep}, and listed in alphabetical order in the reference section. While references are optional, maximum 8 references are permitted for abstract-based submissions. The following arrangements must be used:

% KAO: Use proper quotes and non-breaking space
\subsubsection{References from Journals:} 
Journals must be cited like~\cite{smith1987} or~\cite{michalis2008}. Names of journals can be abbreviated according to the ``International List of Periodical Title Word Abbreviations''. In case of doubt, write names in full.

\subsubsection{References from Books:} 
Books must be cited like~\cite{foerstner2016}.

\subsubsection{References from other Literature:}
Other literature must be cited like~\cite{smith1987rep} and~\cite{smith2000}.

\subsubsection{References from Websites:}
References from the internet must be cited like~\cite{chan2017} and~\cite{maas2017}. Use ofpersistent identifiers such as the Digital Object Identifier (DOI) rather than URLs is strongly advised. In this case last date of visiting the website can be omitted, as the identifier will not change.

\subsubsection{References from Research Data:}
References from research data must be cited like~\cite{dubayah2013}.

\subsubsection{References from Software Projects:}
References to a software project as a high level container including multiple versions of the software must be cited like~\cite{grass2017}.

\subsubsection{References from Software Versions:}
References to a specific software version must be cited like~\cite{grass2015}.

\subsubsection{References from Software Project Add-ons:}
References to a specific software add-on to a software project must be cited like~\cite{lennert2017}.

\subsubsection{References from Software Repository:}
References from software repositories must be cited like~\cite{gago2016}. 

{
	\begin{spacing}{1.17}
		\normalsize
		\bibliography{ISPRSguidelines_authors} % Include your own bibliography (*.bib), style is given in isprs.cls
	\end{spacing}
}

\end{document}

